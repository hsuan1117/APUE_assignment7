% **************************************************
% Document Class Definition
% **************************************************
\documentclass[%
    paper=A4,               % paper size --> A4 is default in Germany
    twoside=false,           % onesite or twoside printing
    openright,              % doublepage cleaning ends up right side
    parskip=full,           % spacing value / method for paragraphs
    chapterprefix=true,     % prefix for chapter marks
    11pt,                   % font size
    headings=normal,        % size of headings
    bibliography=totoc,     % include bib in toc
    listof=totoc,           % include listof entries in toc
    titlepage=on,           % own page for each title page
    captions=tableabove,    % display table captions above the float env
    draft=false,            % value for draft version
]{scrreprt}%


% **************************************************
% Setup YOUR thesis document in this file !
% **************************************************
% !TEX root = book.tex
% !TeX encoding = UTF-8


% **************************************************
% Files' Character Encoding
% **************************************************
\PassOptionsToPackage{utf8}{inputenc}
\usepackage{inputenc}
\usepackage{amsfonts}

% **************************************************
% Load and Configure Packages
% **************************************************
\usepackage[english]{babel} % babel system, adjust the language of the content
\PassOptionsToPackage{% setup clean thesis style
    figuresep=colon,%
    sansserif=false,%
    hangfigurecaption=false,%
    hangsection=true,%
    hangsubsection=true,%
    colorize=full,%
    colortheme=bluemagenta,%
    bibsys=biber,%
    bibfile=bib-refs,%
    bibstyle=alphabetic,%
    wrapfooter=false,%
}{cleanthesis}

\PassOptionsToPackage{table,xcdraw}{xcolor}
\usepackage{cleanthesis}

\hypersetup{% setup the hyperref-package options
    pdftitle={\thesisTitle},    %   - title (PDF meta)
    pdfsubject={\thesisSubject},%   - subject (PDF meta)
    pdfauthor={\thesisName},    %   - author (PDF meta)
    plainpages=false,           %   -
    colorlinks=false,           %   - colorize links?
    pdfborder={0 0 0},          %   -
    breaklinks=true,            %   - allow line break inside links
    bookmarksnumbered=true,     %
    bookmarksopen=true          %
}

\usepackage{epigraph}
\setlength\epigraphwidth{.8\textwidth}
\setlength\epigraphrule{0pt}

\usepackage{tabularx}
\usepackage{graphicx}
\usepackage{pgf,tikz,pgfplots}
\usepackage{mathrsfs}
\pgfplotsset{compat=1.15}
\usetikzlibrary{datavisualization}
\usetikzlibrary{datavisualization.formats.functions}
\usetikzlibrary{arrows,positioning}
\usetikzlibrary{shadows}
\usepackage[bottom]{footmisc}
\makeatletter
\@addtoreset{footnote}{section}
\makeatother
\renewcommand{\thefootnote}{\roman{footnote}}
\usepackage{wrapfig}
\usepackage{xcolor}
\usepackage{listings}
\lstset{
    language=C++, % choose the language of the code
    basicstyle=\footnotesize\ttfamily, %\color{red},
	keywordstyle=\color{blue}\ttfamily,
    stringstyle=\color{red}\ttfamily,
    commentstyle=\color{gray}\ttfamily,
    extendedchars=false,
    numbers=left, % where to put the line-numbers
    numberstyle=\tiny, % the size of the fonts that are used for the line-numbers
    showspaces=false, % show spaces adding particular underscores
    showstringspaces=false, % underline spaces within strings
    showtabs=false, % show tabs within strings adding particular underscores
    frame=single, % adds a frame around the code
    tabsize=2, % sets default tabsize to 2 spaces
    captionpos=b, % sets the caption-position to bottom
    breaklines=true, % sets automatic line breaking
    breakatwhitespace=false,
}

%% packages & settings for xeCJK
\let\latinencoding\relax %% fix build (https://github.com/wspr/fontspec/issues/312)
\usepackage{fontspec}
\usepackage{xeCJK}
\setmainfont{Times New Roman}
\setCJKmainfont[AutoFakeBold=4,AutoFakeSlant=.3]{[微軟正黑體.ttc]}
%\setCJKmainfont[AutoFakeBold=4,AutoFakeSlant=.3]{DFKai-SB}
%\setCJKmainfont[AutoFakeBold=4,AutoFakeSlant=.3]{PingFang TC}
%\setCJKmainfont[AutoFakeBold=4,AutoFakeSlant=.3]{Heiti TC}
\setCJKsansfont{[微軟正黑體.ttc]}
%\setCJKmonofont{Andale Mono}
\XeTeXlinebreaklocale "zh"
\XeTeXlinebreakskip = 0pt plus 1pt
\setlength{\parindent}{2em}

%$ Problem Gen
\usepackage{amsmath}
\usepackage{float}
\newcommand{\problembox}[3]{
    \begin{table}[H]
        \centering
        \begin{tabularx}{\textwidth}{|Xr|}
            \hline
            #1 & #2 \\ \hline
            \multicolumn{2}{|>{\hsize=\dimexpr2\hsize+2\tabcolsep+\arrayrulewidth\relax}X|}{#3} \\ \hline
        \end{tabularx}
    \end{table}
}

\usepackage{outlines}
\usepackage{caption}
\usepackage{subcaption}
\usepackage{algorithm}
\usepackage{algpseudocode}
\usepackage{multirow}
%% For flow.tex
\usepackage{amsthm}
\usepackage{longtable}
\usepackage{array}

\DeclareMathOperator{\ori}{ori}
\DeclareMathOperator{\sign}{sign}

% **************************************************
% Document CONTENT
% **************************************************
\begin{document}

% --------------------------
% rename document parts
% --------------------------
%\renewcaptionname{ngerman}{\figurename}{Abb.}
%\renewcaptionname{ngerman}{\tablename}{Tab.}
\renewcaptionname{english}{\figurename}{圖}
\renewcaptionname{english}{\tablename}{表}
\renewcommand{\lstlistingname}{程式碼}
\makeatletter
\@addtoreset{algorithm}{chapter}% algorithm counter resets every chapter
\makeatother
\renewcommand{\thealgorithm}{\arabic{chapter}.\arabic{algorithm}}
\newtheorem{definition}{定義}[chapter] %[subsection]
\newtheorem{theorem}{定理}[chapter] %[subsection]
\newtheorem{corollary}{推論}[theorem]
\newtheorem{lemma}[theorem]{引理}

\newcommand{\algorithmname}{演算法}
\floatname{algorithm}{\algorithmname}
\newcommand{\theoremname}{定理}
\floatname{theorem}{\theoremname}

\newcommand{\Ord}{\operatorname{\mathcal{O}}}
\newcommand{\ord}[1]{\mathcal{O}\left(#1\right)}
\newcommand{\abs}[1]{\lvert #1 \rvert}
\newcommand{\floor}[1]{\lfloor #1 \rfloor}
\newcommand{\ceil}[1]{\lceil #1 \rceil}
\newcommand{\opord}{\operatorname{\mathcal{O}}}
\newcommand{\argmax}{\operatorname{arg\,max}}
\newcommand{\str}[1]{\texttt{"#1"}}
\newcommand{\tabincell}[2]{\begin{tabular}{@{}#1@{}}#2\end{tabular}}

% for display keyboard
\newcommand*\keystroke[1]{%
  \tikz[baseline=(key.base)]
    \node[%
      draw,
      fill=white,
      drop shadow={shadow xshift=0.25ex,shadow yshift=-0.25ex,fill=black,opacity=0.75},
      rectangle,
      rounded corners=2pt,
      inner sep=1pt,
      line width=0.5pt,
      font=\scriptsize\sffamily
    ](key) {#1\strut}
  ;
}

% --------------------------
% Body matter
% --------------------------
\pagenumbering{arabic}			% arabic page numbering
\setcounter{page}{1}			% set page counter
\pagestyle{maincontentstyle} 	% fancy header and footer

% HERE FOR REPORT %
\chapter{Assignment7 Report}


\section{Implementation Details}

Why the Child Process Does Not Have a Controlling Terminal:

When a new session is created by the child process using setsid(), the child becomes the session leader of this new session. As part of this, the child process also loses its controlling terminal if it had one. This is because sessions are designed to be independent entities in terms of controlling terminals, allowing for background processes and daemons that do not need user interaction.
PID, PGRP, and TPGID Values:

PID (Process ID): Unique identifier for each process.
PGRP (Process Group ID): ID of the process group. A process group is a collection of one or more processes, usually associated with the same job, that can receive signals collectively.
TPGID (Terminal Process Group ID): ID of the foreground process group of the controlling terminal. If a process does not have a controlling terminal, this will be -1.
In the case of our child process:

The PID will be the unique identifier for the child process.
The PGRP will be the same as the child's PID since it becomes a process group leader.
The TPGID will be -1, indicating no controlling terminal.
Compile and run this program in a Unix-like environment to observe the behavior. Make sure to have necessary permissions and configurations to execute system calls like fork() and setsid().

\newpage
\section{Codes}
\lstinputlisting[language=C, caption=assignment7.c]{assignment7.c}

\lstinputlisting[language=shell, caption=指令紀錄]{output.sh}
\newpage
\lstinputlisting[language=makefile, caption=Makefile]{Makefile}


{%
\setstretch{1.1}
\renewcommand{\bibfont}{\normalfont\small}
\setlength{\biblabelsep}{0pt}
\setlength{\bibitemsep}{0.5\baselineskip plus 0.5\baselineskip}
\printbibliography[nottype=online]
\printbibliography[heading=subbibliography,title={Webpages},type=online,prefixnumbers={@}]
}
\cleardoublepage

\end{document}