\chapter{Assignment7 Report}


\section{Some Background}
\begin{enumerate}
\item PID (Process ID): Unique identifier for each process.
\item PGRP (Process Group ID): ID of the process group. A process group is a collection of one or more processes.
\item TPGID (Terminal Process Group ID): ID of the foreground process group of the controlling terminal. If a process does not have a controlling terminal, this will be 0 (in Unix environment, like BSD, Mac), or -1 (in Linux environemnet, tested on Ubuntu).
\end{enumerate}

\section{Implementation}
When we \textit{fork()} a process, the return value of \textit{fork()} is the PID of the child process in the parent process, and 0 in the child process.
So we can use this to distinguish the parent process and the child process.

In the child process, we can use \textit{setsid()} to create a new session.

Because the child process has its own session, so it does not have a controlling terminal, and it'll be the leader of the process group.

The value of PID, PGID will be the same in the child process, and the value of TPGID will be 0 (in Unix environment, like BSD, Mac), or -1 (in Linux environemnet, tested on Ubuntu).

\newpage


\section{Codes}
\lstinputlisting[language=C, caption=assignment7.c]{assignment7.c}
\newpage
\lstinputlisting[language=shell, caption=指令紀錄 (On BSD)]{bsd.sh}
\lstinputlisting[language=shell, caption=指令紀錄 (On Ubuntu)]{ubuntu.sh}
\newpage
\lstinputlisting[language=makefile, caption=Makefile]{Makefile}
